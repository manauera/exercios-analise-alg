\documentclass{article}

\renewcommand{\thesection}{}
\renewcommand{\thesubsection}{\arabic{section}.\arabic{subsection}}
\makeatletter
\def\@seccntformat#1{\csname #1ignore\expandafter\endcsname\csname the#1\endcsname\quad}
\let\sectionignore\@gobbletwo
\let\latex@numberline\numberline
\def\numberline#1{\if\relax#1\relax\else\latex@numberline{#1}\fi}
\makeatother



\usepackage[utf8]{inputenc}

\title{Lista de Exercícios 4}
\author{Gustavo Higuchi}
\date{\today}

\usepackage{natbib}
\usepackage{graphicx}
\usepackage{amssymb}
\usepackage{amsthm}
\usepackage{amsmath}
\usepackage{algorithm2e}
\usepackage{color}   %May be necessary if you want to color links

\usepackage{mathtools}
\DeclarePairedDelimiter\ceil{\lceil}{\rceil}
\DeclarePairedDelimiter\floor{\lfloor}{\rfloor}

% usado para linkar cada section na tabela de conteúdo com a respectiva
% página no documento
\usepackage{hyperref}
\hypersetup{
    colorlinks,
    citecolor=black,
    filecolor=black,
    linkcolor=black,
    urlcolor=black,
    linktoc=all  
}

%o começo do documento
\begin{document}

% compila o título
\maketitle

% compila a tabela de conteúdos
\tableofcontents
\newpage


\chapter{}

\section{Exercício 1}
Informalmente, o custo do algoritmo ficaria $\Theta(n^3)$.

\section{Exercício 2}
\subsection*{(a)}
	\begin{algorithm}
	\KwIn{Um vetor A[1..n]}
	\KwOut{O vetor A ordenado}
	\caption{Selection Sort}
	\For{$i \gets 0$ \textbf{to} $n-1$}{
		$m \gets i$\\
		\For{$j \gets i+1$ \textbf{to} $n$}{
			\If{$A[m] < A[j]$}{
				$m \gets j$
			}
		}
        $aux \gets A[i]$\\
		$A[i] \gets A[m]$\\
        $A[m] \gets aux$
	}
	\end{algorithm}

\subsection*{(b)}
    Como o menor elemento do vetor é sempre trocado com a primeira posição do vetor, 
    quando o algoritmo chega no último elemento, todos os $n-1$ elementos anteriores 
    são menores que o último elemento.

\subsection*{(c)}
    Para este algoritmo, o melhor caso é igual ao pior caso, e é exatamente $(n-1)*\sum\limits_{i=1}^{n-1}(n-i)$

\subsection*{(d)}
    Dado o tempo exato do exercício (c), assintoticamente falando o algoritmo roda em
    tempo $\Theta(n^2)$.
\\
\\
\section{Exercício 3}
\subsection*{(a)}
    \begin{algorithm}
    \KwIn{Uma sequencia de $n$ números $A = (a_1, a_2, .. ,a_n)$ e um valor $v$}
    \KwOut{Um índice tal que $v = A[i]$ ou um valor especial $NIL$ se $v$ não aparece em A}
    \caption{Busca Linear}
    \For{$i \gets 0$ \textbf{to} $n$}{
        \If{$A[i] = v$}{
            \Return $v$
        }
    }
    \Return $NIL$
    \end{algorithm}
\subsection*{(b)}
    

\subsection*{(c)}
    Se o elemento pode estar distribuido estatisticamente semelhante, metade das vezes estará na metade inicial, na outra metade, estará na metade final. Daria uma média 
    de $\dfrac{n}{2}$ comparações

\subsection*{(d)}
    O pior caso seria se o elemento estivesse na última posição do vetor, fazendo $n$ 
    comparações. 

\subsection*{(e)}
    Em ambos os casos, na notação assintótica, teremos um tempo de execução de $\Theta(n)$.
\end{document}