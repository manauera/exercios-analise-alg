\documentclass{article}

\renewcommand{\thesection}{}
\renewcommand{\thesubsection}{\arabic{section}.\arabic{subsection}}
\makeatletter
\def\@seccntformat#1{\csname #1ignore\expandafter\endcsname\csname the#1\endcsname\quad}
\let\sectionignore\@gobbletwo
\let\latex@numberline\numberline
\def\numberline#1{\if\relax#1\relax\else\latex@numberline{#1}\fi}
\makeatother



\usepackage[utf8]{inputenc}

\title{Lista de Exercícios 12}
\author{Gustavo Higuchi}
\date{\today}

\usepackage{natbib}
\usepackage{graphicx}
\usepackage{amssymb}
\usepackage{amsthm}
\usepackage{amsmath}
\usepackage{color}   %May be necessary if you want to color links
\usepackage[portuguese, ruled, linesnumbered]{algorithm2e}

\usepackage{mathtools}
\DeclarePairedDelimiter\ceil{\lceil}{\rceil}
\DeclarePairedDelimiter\floor{\lfloor}{\rfloor}

% usado para linkar cada section na tabela de conteúdo com a respectiva
% página no documento
\usepackage{hyperref}
\hypersetup{
    colorlinks,
    citecolor=black,
    filecolor=black,
    linkcolor=black,
    urlcolor=black,
    linktoc=all
}

%o começo do documento
\begin{document}

% compila o título
\maketitle

% compila a tabela de conteúdos
\tableofcontents
\newpage


\chapter{}
\section{Exercício 1}
\newtheorem{teo1}{Teorema}
\begin{teo1}
    \begin{equation}
        T(n)=\begin{cases}
        1, & \text{se $n = 1$}\\
        T(\floor*{n/2}) + 1, &\text{se $n > 1$}
    \end{cases}
    \end{equation}
    T(n) é $O(\log n)$
\end{teo1}
\begin{proof}
    \hfill \break
    \textbf{Hipótese} : $T(k) \leq c\log k$, para todo $n_0 \leq k < n$, onde $c$ e $n_0$ são constantes\newline
    \textbf{Passo} : quero provar que para $T(n) \leq c\log n$.\newline
    \hspace*{30pt} Assim temos \newline
    \begin{equation}
        \begin{split}
            T(n) &= T(\ceil*{n/2}) + 1 \\
            & \leq c\log\ceil*{n/2} + 1\\
            & = c(\log n-1) + 1\\
            & = c\log n - c + 1\\
            T(n) & \leq c\log n -c + 1\\
            & \leq c\log n \text{, se $c \leq 1$}
        \end{split}
    \end{equation}
    A afirmação é verdadeira para um $c = 1$ e $n_0 \geq 1$
\end{proof}

\section{Exercício 2}
\newtheorem{teo2}{Teorema}
\begin{teo2}
    \begin{equation}
        T(n)=\begin{cases}
        8, & \text{se n = 1}\\
        2T(\floor*{n/2}) + n, & \text{se n > 1}
    \end{cases}
    \end{equation}
    T(n) é $\Theta(n \log n)$.
\end{teo2}
\begin{proof}
    \hfill \break
    \textbf{Hipótese} : $T(k) \leq ck\log k$, para todo $n_0 \leq k < n$, onde $c$ e $n_0$ são constantes\newline
    \textbf{Passo} : Quero provar que para $T(n) \leq cn\log n$.\newline
    \hspace*{30pt} Assim temos \newline
    \begin{equation}
        \begin{split}
            T(n) &= 2T(\floor*{n/2}) + n \\
            & \leq 2c\floor*{n/2} \log \floor*{n/2} + n\\
            & = 2c\floor*{n/2} (\log n - 1) + n\\
            & = 2c\floor*{n/2}\log n - 2c\floor*{n/2} + n \\
            & \leq cn\log n - cn + n\\
            T(n) & \leq cn \log n
        \end{split}
    \end{equation}
    T(n) é $O(n\log n)$ para um $c = 9$ e $n_0 \geq 2$
    \begin{equation}
        \begin{split}
            T(n) &= 2T(\floor*{n/2}) + n \\
            & \geq 2c\floor*{n/2} \log \floor*{n/2} + n\\
            & = 2c\floor*{n/2} (\log n - 1) + n\\
            & = 2c\floor*{n/2}\log n - 2c\floor*{n/2} + n \\
            & \geq cn\log n - cn + n\\
            T(n) & \geq cn \log n
        \end{split}
    \end{equation}
    T(n) é $\Omega(n\log n)$ para um $c = 1$ e um $n_0 \geq 2$\newline
    Portanto, T(n) é $\Theta(n\log n)$
\end{proof}

\section{Exercício 3}
\newtheorem{teo3}{Teorema}
\begin{teo3}
    \begin{equation}
        T(n)=\begin{cases}
        1, & \text{se n = 1}\\
        2T(\floor*{n/2} + 17) + n, & \text{se n > 1}
    \end{cases}
    \end{equation}
    T(n) é $\Theta(n \log n)$.
\end{teo3}
\begin{proof}
    \hfill \break
    \textbf{Hipótese} : $T(k) \leq ck\log k$, para todo $n_0 \leq k < n$, onde $c$ e $n_0$ são constantes\newline
    \textbf{Passo} : Quero provar que para $T(n) \leq cn\log n$.\newline
    \hspace*{30pt} Assim temos \newline
    \begin{equation}
        \begin{split}
            T(n) &= 2T(\floor*{n/2} + 17) + n \\
            & \leq 2c(\floor*{n/2}+17)\log (\floor*{n/2} + 17) + n\\
            & = 2c(\floor*{n/2}+17)\log \floor*{n/2} + \log(1+\dfrac{17}{\floor*{n/2}}) + n\\
            T(n) & \leq cn \log n
        \end{split}
    \end{equation}
    T(n) é $O(n\log n)$ para um $c = 9$ e $n_0 \geq 2$
    \begin{equation}
        \begin{split}
            T(n) &= 2T(\floor*{n/2}) + n \\
            & \geq 2c\floor*{n/2} \log \floor*{n/2} + n\\
            & = 2c\floor*{n/2} (\log n - 1) + n\\
            & = 2c\floor*{n/2}\log n - 2c\floor*{n/2} + n \\
            & \geq cn\log n - cn + n\\
            T(n) & \geq cn \log n
        \end{split}
    \end{equation}
    T(n) é $\Omega(n\log n)$ para um $c = 1$ e um $n_0 \geq 2$\newline
    Portanto, T(n) é $\Theta(n\log n)$
\end{proof}

\section{Exercício 4}
\subsection*{(a)}
\begin{algorithm}[H]
  \SetAlgoLined
  \Entrada{$A$, $v$}
  \Saida{índice $i$ de $v$ em $A$ }
  \Inicio{
    \Se{$v = A[n/2]$}{
        \Retorna $n/2$
    }
    \Se{$n = 0$}{
        \Retorna $nil$
    }
    \Senao{
        \Se{ $v < A[n/2]$}{
            $busca(A[0,n/2-1], v)$
        }
        \Senao{
            $busca(A[n/2+1, n], v)$
        }
    }
  }
  \label{alg1}
  \caption{busca}
\end{algorithm}

\subsection*{(b)}
\begin{equation}
    busca(A[0,n],v) = \begin{cases}
        n/2&\text{, se A[n/2] = v}\\
        busca(A[0,n/2-1]) + 1&\text{, $se A[n/2] > v$}\\
        busca([n/2+1, n]) + 1&\text{, $se A[n/2] < v$}\\
        nil&\text{, se n = 0}
    \end{cases}
\end{equation}
    Onde $i = \floor*{\dfrac{inicio+fim}{2}}$

\subsection*{(c)}
\newtheorem{teo4}{Teorema}
\begin{teo4}
    $busca(A, v)$ é $\Theta(\log n)$.
\end{teo4}
\begin{proof}
    \hfill \break
    \textbf{Hipótese} : $busca(A[0,k], v) \leq c\log k$, para todo $n_0 \leq k < n$, onde $n$ é o tamanho de A, $c$ e $n_0$ são constantes\newline
    \textbf{Passo} : Quero provar que para $busca(A[0,n], v) \leq c\log n$.\newline
    \hspace*{30pt}
    \begin{equation}
        \begin{split}
        busca(n, v) & = busca(n/2, v) + 1\\
        & \leq c\log n/2 + 1\\
        & = c\log n
        \end{split}
    \end{equation}
    Então $busca(n)$ é $O(\log n)$, para um $c \geq 1$ e um $n_0 \geq 1$
    \begin{equation}
        \begin{split}
            busca(n, v) & = busca(n/2, v) + 1\\
            & \leq c\log n/2 + 1\\
            & = c\log n
        \end{split}
    \end{equation}
    Então $busca(n)$ é $\Theta(\log n)$, para um $c \geq 1$ e um $n_0 \geq 1$
\end{proof}

\end{document}
