\documentclass{article}

\renewcommand{\thesection}{}
\renewcommand{\thesubsection}{\arabic{section}.\arabic{subsection}}
\makeatletter
\def\@seccntformat#1{\csname #1ignore\expandafter\endcsname\csname the#1\endcsname\quad}
\let\sectionignore\@gobbletwo
\let\latex@numberline\numberline
\def\numberline#1{\if\relax#1\relax\else\latex@numberline{#1}\fi}
\makeatother



\usepackage[utf8]{inputenc}

\title{Lista de Exercícios 4}
\author{Gustavo Higuchi}
\date{\today}

\usepackage{natbib}
\usepackage{graphicx}
\usepackage{amssymb}
\usepackage{amsthm}
\usepackage{amsmath}
\usepackage{color}   %May be necessary if you want to color links

\usepackage{mathtools}
\DeclarePairedDelimiter\ceil{\lceil}{\rceil}
\DeclarePairedDelimiter\floor{\lfloor}{\rfloor}

% usado para linkar cada section na tabela de conteúdo com a respectiva
% página no documento
\usepackage{hyperref}
\hypersetup{
    colorlinks,
    citecolor=black,
    filecolor=black,
    linkcolor=black,
    urlcolor=black,
    linktoc=all
}

%o começo do documento
\begin{document}

% compila o título
\maketitle

% compila a tabela de conteúdos
\tableofcontents
\newpage


\chapter{}

\section{Exercício 1}
Para mostrar que $\log_b n \leq c * \log_a n$, basta
\begin{equation}
    \begin{split}
    \log_b n  = \dfrac{\log_a n}{\log_a b} & \leq c*\log_a n\\
    \dfrac{1}{\log_a b} & \leq c
    \end{split}
\end{equation}

Então, para um $c \geq \dfrac{1}{\log_a b}$ e um $n_0 \geq 1$
\begin{equation}
    \log_b n = O(\log_a n)
\end{equation}

\section{Exercício 2}
Pela definição, $\exists c_1, n_1$ tal que $\bar{f}(n) \leq c_1f(n)$
para $n \geq n_1$\\
E pela definição, $\exists c_2, n_2$ tal que $\bar{g}(n) \leq c_2g(n)$
para $n \geq n_2$\\
Assim\\
\begin{equation}
    \begin{split}
        \bar{f}(n)\bar{g}(n) &\leq c_1f(n)c_2g(n)\\
        &\leq c_3f(n)g(n)\text{, onde }c_3 = c_1 * c_2
    \end{split}
\end{equation}
para $n \geq max\{n_1,n_2\}$

\section{Exercício 3}
Se dividir a função $f(x)$ em 3 funções, teremos
\begin{equation}
    \begin{split}
        g(x) & = x + 1\\
        h(x) & = \log(x^2 + 1)\\
        i(x) & = 3x^2\\
        f(x) & = g(x)h(x) + i(x)
    \end{split}
\end{equation}
Então,
\begin{equation}
    \begin{split}
        g(x) & = O(x + 1) = O(x)\\
        h(x) & = O(\log(x^2 + 1)) = O(\log x^2) = O(\log x)\\
        i(x) & = O(3x^2) = O(x^2)
    \end{split}
\end{equation}

\begin{equation}
    \begin{split}
        f(x) & = O(x)*O(\log x) + O(x^2)\\
        & = O(x\log x) + O(x^2)\\
        & = O(x^2)
    \end{split}
\end{equation}

\section{Exercício 4}
Sim para o primeiro, pois existe uma constante $c$ tal que
\begin{equation}
    2^{n+1} \leq c * 2^n
\end{equation}
\hspace*{40pt}para um $n_0 \geq 1$ e $c = 2$

Porém, para o segundo, temos
\begin{equation}
    2^{2n} = c * 2^n
\end{equation}

para um $n_0 \geq 0$ e um $c = 2^{10000}$, quando $n = 10001$ teremos
\begin{equation}
    \begin{split}
        2^{2*10001} &\leq 2^{10000} * 2^{10001}\\
        2^{20002} & \leq 2^{20001}
    \end{split}
\end{equation}
o que é falso, portanto não há uma constante que multiplica $2^n$ tal que
para todo $n > n_0$, $2^{2n}$ seja menor que $c*2^n$

\section{Exercício 5}
\subsection*{(a)}
\begin{proof}
    \hfill \break
    $2^n = 2 * 2 * 2 * 2 * .. * 2$, enquanto\\
    $n!  = n * (n - 1) * (n - 2) * .. * 4 * 3 * 2 * 1$\\

    Então, para um $n_0 = 1$ e um $c = 2$, $2^n = O(n!)$
\end{proof}

\subsection*{(b)}
\begin{proof}
    \hfill \break
    Como $\log n!$ pode ser reescrito da seguinte forma:
    \begin{equation}
        \log n! = \sum\limits_{i = 0}^{n}\log(n - i)
    \end{equation}
    E assim teremos o seguinte
    \begin{equation}
        \begin{split}
        \sum\limits_{i=0}^{n}log(n-i) & < \sum\limits_{i=0}^{n}\log n\\
        & =n * \log n = O(n\log n)
    \end{split}
    \end{equation}
\end{proof}

\end{document}
