\documentclass{article}

\renewcommand{\thesection}{}
\renewcommand{\thesubsection}{\arabic{section}.\arabic{subsection}}
\makeatletter
\def\@seccntformat#1{\csname #1ignore\expandafter\endcsname\csname the#1\endcsname\quad}
\let\sectionignore\@gobbletwo
\let\latex@numberline\numberline
\def\numberline#1{\if\relax#1\relax\else\latex@numberline{#1}\fi}
\makeatother



\usepackage[utf8]{inputenc}

\title{Lista de Exercícios 4}
\author{Gustavo Higuchi}
\date{\today}

\usepackage{natbib}
\usepackage{graphicx}
\usepackage{amssymb}
\usepackage{amsthm}
\usepackage{amsmath}
\usepackage{color}   %May be necessary if you want to color links
\usepackage[portuguese, ruled, linesnumbered]{algorithm2e}

\usepackage{mathtools}
\DeclarePairedDelimiter\ceil{\lceil}{\rceil}
\DeclarePairedDelimiter\floor{\lfloor}{\rfloor}

% usado para linkar cada section na tabela de conteúdo com a respectiva
% página no documento
\usepackage{hyperref}
\hypersetup{
    colorlinks,
    citecolor=black,
    filecolor=black,
    linkcolor=black,
    urlcolor=black,
    linktoc=all
}

%o começo do documento
\begin{document}

% compila o título
\maketitle

% compila a tabela de conteúdos
\tableofcontents
\newpage


\chapter{}
\section{Exercício 1}
\subsection*{(a)}
O algoritmo sempre executa o mesmo número de passos
\subsection*{(b)}
\begin{equation}
    g(n)=\begin{cases}
    n, & \text{se $n\leq 1$}.\\
    5g(n-1) - 6g(n-2), & \text{caso contrário}.
\end{cases}
\end{equation}

\section{Exercício 2}
\subsection*{(a)}
O algoritmo sempre executa o mesmo número de passos
\subsection*{(b)}
\begin{equation}
    mult(y,z)=\begin{cases}
    0, & \text{se $z = 0$}.\\
    mult(2y, \floor*{\dfrac{z}{2}}) + y(\text{z mod 2}), & \text{caso contrário}.
\end{cases}
\end{equation}

\section{Exercício 3}
\subsection*{(a)}
O algoritmo sempre executa o mesmo número de passos
\subsection*{(b)}
\begin{equation}
    power(y,z)=\begin{cases}
    1, & \text{se }z = 0\\
    power(y^2, \floor*{\dfrac{z}{2}})*y, & \text{se $z$ for ímpar}\\
    power(y^2, \floor*{\dfrac{z}{2}}), & \text{se $z$ for par}
\end{cases}
\end{equation}

\section{Exercício 4}
\subsection*{(a)}
O algoritmo sempre executa o mesmo número de passos, como o algoritmo tem que somar
todos os elementos de uma lista, sempre executará um número de passos igual ao
tamanho da lista.
\subsection*{(b)}
\begin{equation}
    sum(A,n)=\begin{cases}
    A[1], & \text{se $n \leq 1$}\\
    sum(A, n-1)+ A[n], &\text{caso contrário}
\end{cases}
\end{equation}

\section{Exercício 5}

\subsection*{(a)}
O algoritmo sempre executa o mesmo número de passos, como tem que retornar o
\textit{maior} número de uma lista, não tem mágica, tem comparar com todo mundo
ou o algoritmo está errado.
\subsection*{(b)}
\begin{equation}
    maximo(A,x,y)=\begin{cases}
    max(A[x],A[y]), &\text{se $y-x\leq1$}\\
    max(maximo(x,\floor*{(x+y)/2}), maximo(\floor*{(x+y)/2},y)+1, & \text{caso contrário}
\end{cases}
\end{equation}
\end{document}
