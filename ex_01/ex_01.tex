\documentclass{article}

\renewcommand{\thesection}{}
\renewcommand{\thesubsection}{\arabic{section}.\arabic{subsection}}
\makeatletter
\def\@seccntformat#1{\csname #1ignore\expandafter\endcsname\csname the#1\endcsname\quad}
\let\sectionignore\@gobbletwo
\let\latex@numberline\numberline
\def\numberline#1{\if\relax#1\relax\else\latex@numberline{#1}\fi}
\makeatother

\usepackage[utf8]{inputenc}

\title{Lista de Exercícios 01}
\author{Gustavo Higuchi}
\date{Agosto 2016}

\usepackage{natbib}
\usepackage{graphicx}
\usepackage{amssymb}
\usepackage{amsmath}
\usepackage{color}   %May be necessary if you want to color links

% usado para linkar cada section na tabela de conteúdo com a respectiva
% página no documento
\usepackage{hyperref}
\hypersetup{
    colorlinks,
    citecolor=black,
    filecolor=black,
    linkcolor=black,
    urlcolor=black,
    linktoc=all  
}

%o começo do documento
\begin{document}

% compila o título
\maketitle

% compila a tabela de conteúdos
\tableofcontents
\newpage


\chapter{}
\section{Exercício 1}
\subsection*{(a)}

\begin{equation} \label{eq1}
\begin{split}
\sum\limits_{k=1}^{5}(k_i + 1) & =  (1 + 1) + \sum\limits_{k=2}^{5}(k + 1) \\
& =  2 + (2 + 1) + \sum\limits_{k=3}^{5}(k + 1) \\
& =  5 + (3 + 1) + \sum\limits_{k=4}^{5}(k + 1) \\
& =  9 + (4 + 1) + \sum\limits_{k=5}^{5}(k + 1) \\
& =  14 + (5 + 1)  \\
& =  20  \\
\end{split}
\end{equation}

\subsection*{(b)}
\begin{equation} \label{eq2}
\begin{split}
\sum\limits_{j=0}^{4}(-2)^j & =  (-2)^0 + \sum\limits_{j=1}^{4}(-2)^j \\
& =  -2 + (-2)^1 + \sum\limits_{j=1}^{4}(-2)^j \\
& =  -4 + (-2)^2 + \sum\limits_{j=2}^{4}(-2)^j \\
& =  0 + (-2)^3 + \sum\limits_{j=3}^{4}(-2)^j \\
& =  -8 + (-2)^4 + \sum\limits_{j=4}^{4}(-2)^j \\
& = -8 + 16 = 8\\
\end{split}
\end{equation}

\subsection*{(c)}
\begin{equation} \label{eq3}
\begin{split}
\sum\limits_{t=1}^{100}3 & =  3 + \sum\limits_{t=2}^{100}3 \\
& = 6 + \sum\limits_{t=3}^{100}3 \\
& = 9 + \sum\limits_{t=4}^{100}3 \\
& ...
& = 297 + \sum\limits_{t=100}^{100}3 \\
& = 300 \\
\end{split}
\end{equation}


\subsection*{(d)}
\begin{equation} \label{eq4}
\begin{split}
\sum\limits_{j=0}^{8}(2^{j+1} - 2^j) & =  2^{0+1} + (-2)^0 + \sum\limits_{t=1}^{8}(2^{j+1} - 2^j) \\
& =  2 + 2^{1+1} + (-2)^1 + \sum\limits_{t=1}^{8}(2^{j+1} - 2^j) \\
& =  4 + 2^{2+1} + (-2)^2 + \sum\limits_{t=2}^{8}(2^{j+1} - 2^j) \\
& =  16 + 2^{3+1} + (-2)^3 + \sum\limits_{t=3}^{8}(2^{j+1} - 2^j) \\
& =  28 + 2^{4+1} + (-2)^4 + \sum\limits_{t=4}^{8}(2^{j+1} - 2^j) \\
& =  76 + 2^{5+1} + (-2)^5 + \sum\limits_{t=5}^{8}(2^{j+1} - 2^j) \\
& =  108 + 2^{6+1} + (-2)^6 + \sum\limits_{t=6}^{8}(2^{j+1} - 2^j) \\
& =  300 + 2^{7+1} + (-2)^7 + \sum\limits_{t=7}^{8}(2^{j+1} - 2^j) \\
& =  428 + 2^{8+1} + (-2)^8 + \sum\limits_{t=8}^{8}(2^{j+1} - 2^j) \\
& =  256+512+428 = 1196
\end{split}
\end{equation}


\subsection*{(e)}
\begin{equation} \label{eq5}
\begin{split}
\sum\limits_{i=1}^{2}\sum\limits_{j=1}^{3}(i + j) & =  1 + 1 + \sum\limits_{i=1}^{2}\sum\limits_{j=2}^{3}(i + j) \\
& =  2 + 1 + 2 + \sum\limits_{i=1}^{2}\sum\limits_{j=3}^{3}(i + j) \\
& =  5 + 1 + 3 + \sum\limits_{i=2}^{2}\sum\limits_{j=1}^{3}(i + j) \\
& =  9 + 2 + 1 + \sum\limits_{i=2}^{2}\sum\limits_{j=2}^{3}(i + j) \\
& =  12 + 2 + 2 + \sum\limits_{i=2}^{2}\sum\limits_{j=3}^{3}(i + j) \\
& =  16 + 2 + 3 = 21\\
\end{split}
\end{equation}

\subsection*{(f)}
\begin{equation} \label{eq6}
\begin{split}
\sum\limits_{i=0}^{2}\sum\limits_{j=0}^{3}(2i + 3j) & =  0 + 0 + \sum\limits_{i=0}^{2}\sum\limits_{j=1}^{3}(2i + 3j) \\
& =  0 + 3 + \sum\limits_{i=0}^{2}\sum\limits_{j=2}^{3}(2i + 3j) \\
& =  3 + 0 + 6 + \sum\limits_{i=0}^{2}\sum\limits_{j=3}^{3}(2i + 3j) \\
& =  9 + 0 + 9 + \sum\limits_{i=1}^{2}\sum\limits_{j=0}^{3}(2i + 3j) \\
& =  18 + 2 + 0 + \sum\limits_{i=1}^{2}\sum\limits_{j=1}^{3}(2i + 3j) \\
& =  20 + 2 + 3 + \sum\limits_{i=1}^{2}\sum\limits_{j=2}^{3}(2i + 3j) \\
& =  25 + 2 + 6 + \sum\limits_{i=1}^{2}\sum\limits_{j=3}^{3}(2i + 3j) \\
& =  33 + 2 + 9 + \sum\limits_{i=2}^{2}\sum\limits_{j=0}^{3}(2i + 3j) \\
& =  44 + 4 + 0 + \sum\limits_{i=2}^{2}\sum\limits_{j=1}^{3}(2i + 3j) \\
& =  48 + 4 + 3 + \sum\limits_{i=2}^{2}\sum\limits_{j=2}^{3}(2i + 3j) \\
& =  55 + 4 + 6 + \sum\limits_{i=2}^{2}\sum\limits_{j=3}^{3}(2i + 3j) \\
& =  65 + 4 + 9 = 78 \\
\end{split}
\end{equation}


\subsection*{(g)}
\begin{equation} \label{eq7}
\begin{split}
\sum\limits_{i=1}^{3}\sum\limits_{j=0}^{2}i & =  1 + \sum\limits_{i=1}^{3}\sum\limits_{j=1}^{2}i \\
& =  1 + 1 + \sum\limits_{i=1}^{3}\sum\limits_{j=2}^{2}i \\
& =  2 + 1 + \sum\limits_{i=2}^{3}\sum\limits_{j=0}^{2}i \\ 
& =  3 + 2 + \sum\limits_{i=2}^{3}\sum\limits_{j=1}^{2}i \\ 
& =  5 + 2 + \sum\limits_{i=2}^{3}\sum\limits_{j=2}^{2}i \\ 
& =  7 + 2 + \sum\limits_{i=3}^{3}\sum\limits_{j=0}^{2}i \\ 
& =  9 + 3 + \sum\limits_{i=3}^{3}\sum\limits_{j=1}^{2}i \\ 
& =  12 + 3 + \sum\limits_{i=3}^{3}\sum\limits_{j=2}^{2}i \\ 
& =  15 + 3 = 18\\ 
\end{split}
\end{equation}

\subsection*{(h)}
\begin{equation} \label{eq8}
\begin{split}
\sum\limits_{i=1}^{3}\sum\limits_{j=0}^{2}j & =  0 + \sum\limits_{i=1}^{3}\sum\limits_{j=1}^{2}j \\
& =  0 + 1 + \sum\limits_{i=1}^{3}\sum\limits_{j=2}^{2}j \\
& =  1 + 2 + \sum\limits_{i=2}^{3}\sum\limits_{j=0}^{2}j \\ 
& =  3 + 0 + \sum\limits_{i=2}^{3}\sum\limits_{j=1}^{2}j \\ 
& =  3 + 1 + \sum\limits_{i=2}^{3}\sum\limits_{j=2}^{2}j \\ 
& =  4 + 2 + \sum\limits_{i=3}^{3}\sum\limits_{j=0}^{2}j \\ 
& =  6 + 0 + \sum\limits_{i=3}^{3}\sum\limits_{j=1}^{2}j \\ 
& =  6 + 1 + \sum\limits_{i=3}^{3}\sum\limits_{j=2}^{2}j \\ 
& =  7 + 2 = 9\\ 
\end{split}
\end{equation}

\section{Exercício 4}
\subsection*{(a)}
\begin{equation}
	\log_2 1024 = 10
\end{equation}

\subsection*{(b)}
\begin{equation}
	\log_{10} 0.0001 = -3
\end{equation}

\subsection*{(c)}
\begin{equation}
	\log_{49} 7 = \dfrac{1}{2} 
\end{equation}

\subsection*{(d)}
\begin{equation}
	\log_{32} \dfrac{1}{4}  = -\dfrac{2}{5} 
\end{equation}

\section{Exercício 5}
\subsection*{(a)}
\begin{equation}
	\log_5 125  = 3
\end{equation}

\subsection*{(b)}
\begin{equation}
	\log_{81} 3  = \dfrac{1}{3}
\end{equation}

\subsection*{(c)}
\begin{equation}
	\log_e (\dfrac{1}{e^3})  = 3
\end{equation}

\subsection*{(d)}
\begin{equation}
	\log_c \sqrt{c} = \dfrac{1}{2} 
\end{equation}

\section{Exercício 6}]
\subsection*{(a)}
\begin{equation}
	\log_2 x = 2y
\end{equation}

\subsection*{(b)}
\begin{equation}
	\log_8 x = \dfrac{1}{2}y
\end{equation}

\subsection*{(c)}
\begin{equation}
	\log_{16} x = \dfrac{1}{4}y
\end{equation}

\end{document}

