\documentclass{article}

\renewcommand{\thesection}{}
\renewcommand{\thesubsection}{\arabic{section}.\arabic{subsection}}
\makeatletter
\def\@seccntformat#1{\csname #1ignore\expandafter\endcsname\csname the#1\endcsname\quad}
\let\sectionignore\@gobbletwo
\let\latex@numberline\numberline
\def\numberline#1{\if\relax#1\relax\else\latex@numberline{#1}\fi}
\makeatother

\usepackage[utf8]{inputenc}

\title{Lista de Exercícios 2}
\author{Gustavo Higuchi}
\date{\today}

\usepackage{natbib}
\usepackage{graphicx}
\usepackage{amssymb}
\usepackage{amsthm}
\usepackage{amsmath}
\usepackage{color}   %May be necessary if you want to color links

% usado para linkar cada section na tabela de conteúdo com a respectiva
% página no documento
\usepackage{hyperref}
\hypersetup{
    colorlinks,
    citecolor=black,
    filecolor=black,
    linkcolor=black,
    urlcolor=black,
    linktoc=all  
}

%o começo do documento
\begin{document}

% compila o título
\maketitle

% compila a tabela de conteúdos
\tableofcontents
\newpage


\chapter{}

\section{Exercício 1}

\subsection*{(a)}
\newtheorem{teo1}{Teorema}
\begin{teo1}
    Para qualquer $n \in  \mathbb{Z^+}$, a seguinte equação é verdadeira.
	\begin{equation}
		1^3 + 2^3 + .. + n^3 = \dfrac{n^2(n+1)^2}{4}
	\end{equation}	
\end{teo1}


\begin{proof}
	Para n = 1, isso é verdade?
\begin{equation}
\begin{split}
	1^3 & = \dfrac{1^2(1+1)^2}{4} \\
	1 & = \dfrac{1(2)^2}{4} \\
	1 & = \dfrac{4}{4} = 1
\end{split}
\end{equation}

	Sim! É verdade. E para n = 2?
\begin{equation}
\begin{split}
	1^3 + 2^3 & = \dfrac{2^2(2+1)^2}{4} \\
	1 + 8& = \dfrac{2^2(2+1)^2}{4} \\
	9 & = \dfrac{4(9)}{4} = 9
\end{split}
\end{equation}

	Vou supor que para um n = k dá certo também, então para $n = k + 1$ dá certo?
\begin{equation}
\begin{split}
	1^3 + 2^3 + .. + k^3 + (k+1)^3 = \dfrac{(k+1)^2((k+1)+1)^2}{4} \\
	\dfrac{k^2(k+1)^2}{4} + (k+1)^3 = \dfrac{ (k+1)^2(k+2)^2 }{ 4 } \\
	\dfrac{k^2(k+1)^2 + 4(k+1)^3}{4}  = \dfrac{ (k+1)^2(k+2)^2 }{ 4 } \\
	\dfrac{ (k+1)^2(k^2 + 4(k+1)) }{4}  = \dfrac{ (k+1)^2(k+2)^2 }{ 4 } \\
	\dfrac{ (k+1)^2(k^2 + 4k + 4) }{4}  = \dfrac{ (k+1)^2(k+2)^2 }{ 4 } \\
	\dfrac{ (k+1)^2(k + 2)^2 }{4}  = \dfrac{ (k+1)^2(k+2)^2 }{ 4 } 
\end{split}
\end{equation}

	Dá! Sucesso! Provado por indução!
\end{proof}


\subsection*{(b)}
\newtheorem{teo2}{Teorema}
\begin{teo2}
    Para qualquer $n \in  \mathbb{N}$ e todo $c \in \mathbb{C} - {1,0}$,
    a seguinte equação é verdadeira.
	\begin{equation}
		\sum\limits_{i=0}^{n} c^i = \dfrac{c^{n+1} - 1}{c - 1}
	\end{equation}	
\end{teo2}
\begin{proof}
	Para $n = 1$ e $c = 2$, temos
	\begin{equation}
	\begin{split}
		\sum\limits_{i=0}^{1} c^i = c^0 + c^1 & = \dfrac{c^{1+1} - 1}{c - 1} \\
		1 + c^1 & = \dfrac{c^2 - 1}{c - 1} \\ 
		1 + 2 & = \dfrac{2^2 - 1}{2 - 1} \\
		3 & = 3
	\end{split}
	\end{equation}

	É verdade, daí para $n = 2$ e $c = 3$, temos
	\begin{equation}
	\begin{split}
		\sum\limits_{i=0}^{2} c^i = c^0 + c^1 + c^2 & = \dfrac{c^{2+1} - 1}{c - 1} \\
		1 + 3 + 3^2 & = \dfrac{ 3^3 - 1}{3 - 1}\\
		13 & = \dfrac{26}{2} = 13
	\end{split}
	\end{equation}

	Então, vamos supor que para $n = k$ e $c = l$ isso também vale, assim temos

	\begin{equation}
		\begin{split}
		\sum\limits_{i=0}^{k} c^i = c^0 + c^1 + .. + c^k & = \dfrac{c^{k+1} - 1}{c - 1} \\
		\dfrac{(c^0+c^k)k}{2} & = \dfrac{c^{k+1} - 1}{c - 1} \\
		\dfrac{(c^k + 1)}{2} & = \dfrac{c^{k+1} - 1}{c - 1}\\
		c^k + 1 & = c^{k+1} - 1
		\end{split}
	\end{equation}
\end{proof}
\end{document}
