\documentclass{article}

\renewcommand{\thesection}{}
\renewcommand{\thesubsection}{\arabic{section}.\arabic{subsection}}
\makeatletter
\def\@seccntformat#1{\csname #1ignore\expandafter\endcsname\csname the#1\endcsname\quad}
\let\sectionignore\@gobbletwo
\let\latex@numberline\numberline
\def\numberline#1{\if\relax#1\relax\else\latex@numberline{#1}\fi}
\makeatother

\usepackage[utf8]{inputenc}

\title{Lista de Exercícios 2}
\author{Gustavo Higuchi}
\date{\today}

\usepackage{natbib}
\usepackage{graphicx}
\usepackage{amssymb}
\usepackage{amsthm}
\usepackage{amsmath}
\usepackage{color}   %May be necessary if you want to color links

% usado para linkar cada section na tabela de conteúdo com a respectiva
% página no documento
\usepackage{hyperref}
\hypersetup{
    colorlinks,
    citecolor=black,
    filecolor=black,
    linkcolor=black,
    urlcolor=black,
    linktoc=all  
}

%o começo do documento
\begin{document}

% compila o título
\maketitle

% compila a tabela de conteúdos
\tableofcontents
\newpage


\chapter{}

\section{Exercício 1}

\subsection*{(a)}
\newtheorem{teo1}{Teorema}
\begin{teo1}
    Para qualquer $n \in  \mathbb{Z^+}$, a seguinte equação é verdadeira.
	\begin{equation}
		1^3 + 2^3 + .. + n^3 = \dfrac{n^2(n+1)^2}{4}
	\end{equation}	
\end{teo1}


\begin{proof}
	Para n = 1, isso é verdade?
\begin{equation}
\begin{split}
	1^3 & = \dfrac{1^2(1+1)^2}{4} \\
	1 & = \dfrac{1(2)^2}{4} \\
	1 & = \dfrac{4}{4} = 1
\end{split}
\end{equation}

	Sim! É verdade. E para n = 2?
\begin{equation}
\begin{split}
	1^3 + 2^3 & = \dfrac{2^2(2+1)^2}{4} \\
	1 + 8& = \dfrac{2^2(2+1)^2}{4} \\
	9 & = \dfrac{4(9)}{4} = 9
\end{split}
\end{equation}

	Vou supor que para um n = k dá certo também, então para $n = k + 1$ dá certo?
\begin{equation}
\begin{split}
	1^3 + 2^3 + .. + k^3 + (k+1)^3 = \dfrac{(k+1)^2((k+1)+1)^2}{4} \\
	\dfrac{k^2(k+1)^2}{4} + (k+1)^3 = \dfrac{ (k+1)^2(k+2)^2 }{ 4 } \\
	\dfrac{k^2(k+1)^2 + 4(k+1)^3}{4}  = \dfrac{ (k+1)^2(k+2)^2 }{ 4 } \\
	\dfrac{ (k+1)^2(k^2 + 4(k+1)) }{4}  = \dfrac{ (k+1)^2(k+2)^2 }{ 4 } \\
	\dfrac{ (k+1)^2(k^2 + 4k + 4) }{4}  = \dfrac{ (k+1)^2(k+2)^2 }{ 4 } \\
	\dfrac{ (k+1)^2(k + 2)^2 }{4}  = \dfrac{ (k+1)^2(k+2)^2 }{ 4 } 
\end{split}
\end{equation}

	Dá! Sucesso! Provado por indução!
\end{proof}


\subsection*{(b)}
\newtheorem{teo2}{Teorema}
\begin{teo2}
    Para qualquer $n \in  \mathbb{N}$ e todo $c \in \mathbb{C} - {1,0}$,
    a seguinte equação é verdadeira.
	\begin{equation}
		\sum\limits_{i=0}^{n} c^i = \dfrac{c^{n+1} - 1}{c - 1}
	\end{equation}	
\end{teo2}
\begin{proof}
	Para $n = 0$, temos
	\begin{equation}
	\begin{split}
		\sum\limits_{i=0}^{0} c^i & = \dfrac{c^{0+1} - 1}{c - 1} \\
		c + 1 & = \dfrac{c^2 - 1}{c - 1} \\ 
		(c + 1)(c - 1) & = c^2 - 1 \\
		c^2 - c + c - 1 & = \\ 
		c^2 - 1 & = c^2 - 1
	\end{split}
	\end{equation}

	Para $n = 1$, temos
	\begin{equation}
	\begin{split}
		\sum\limits_{i=0}^{1} c^i & = \dfrac{c^{1+1} - 1}{c - 1} \\
		c^2 + c + 1 & = \dfrac{c^3 - 1}{c - 1} \\ 
		(c^2 + c + 1)(c - 1) & = c^2 - 1 \\
		c^3 - c^2 + c^2 -c + c - 1 & = \\ 
		c^2 - 1 & = c^2 - 1
	\end{split}
	\end{equation}

	É verdade, então vamos supor que o teorema é válido para $n = k$, assim

	\begin{equation}
	\begin{split}
		\sum\limits_{i=0}^{k} c^i & = \dfrac{c^{k+1} - 1}{c - 1} \\
	\end{split}
	\end{equation}

	Então, vamos verificar se para $n = k+1$ isso também vale, temos

	\begin{equation}
		\begin{split}
		\sum\limits_{i=0}^{k+1}c^i & = \dfrac{c^{k+2} - 1}{c - 1} \\
		\sum\limits_{i=0}^{k}c^i c^{k+1} & = \dfrac{c^{k+2} - 1}{c - 1} \\
		\dfrac{c^{k+1} - 1}{c - 1} + ( c^{k+1} )(c - 1) & = \dfrac{c^{k+2} - 1}{c - 1}\\
		\dfrac{c^{k+1} - 1 + c^{k+2} - c^{k+1}}{c - 1}  & = \dfrac{c^{k+2} - 1}{c - 1}\\
		\dfrac{c^{k+2} - 1 }{c - 1}  & = \dfrac{c^{k+2} - 1}{c - 1}\\
		\end{split}
	\end{equation}

	É verdade também
\end{proof}

\subsection*{(c)}
\newtheorem{teo3}{Teorema}
\begin{teo3}
	$n^3 + 2n$ é divisível por 3 para todo $n \in \mathbb{N^+}$
\end{teo3}
\begin{proof}
	Para $n = 0$, temos
	\begin{equation}
		\dfrac{0^3 + 2*0}{3} = 0
	\end{equation}

	Que é divisível por 3.

	Para $n = 1$, temos
	\begin{equation}
		\dfrac{1^3 + 2*1}{3} = \dfrac{3}{3} = 1
	\end{equation}

	Também.

	Assim, para um $n = k$, vamos assumir que seja verdade. 

	Então vamos provar pra $n = k + 1$, daí

	\begin{equation}
		\begin{split}
		(k+1)^3 + 2(k + 1) & = (k + 1)(k + 1)(k + 1) + 2k + 2\\
		& = (k^2 + 2k + 1)(k + 1) + 2k + 2\\
		& = (k^3 + k^2 + 2k^2 + 2k + k + 1) + 2k + 2\\
		& = k^3 + 2k + 3k^2 + 3k + 3\\
		& = k^3 + 2k + 3(k^2 + k + 1)
		\end{split}
	\end{equation}

	Então, pela hipótese, $k^3 + 2k$ é divisível por 3 e $3(k^2 + k + 1)$ também.
\end{proof}

\subsection*{(d)}
\newtheorem{teo4}{Teorema}
\begin{teo4}
	$9^n - 1$ é divisível por 8 para qualquer $n \in \mathbb{N^+}$ 
\end{teo4}
\begin{proof}
	Para $n = 0$, temos
	\begin{equation}
		\dfrac{9^0 - 1}{8} = 0
	\end{equation}

	Para $n = 1$, temos
	\begin{equation}
		\dfrac{9^1 - 1}{8} = \dfrac{8}{8} = 1
	\end{equation}

	Então vamos supor que para um $n = k$ isso também dá certo.
	\begin{equation}
		9^k - 1 \text{, é divisível por 8}
	\end{equation}

	Daí, vamos provar para $n = k + 1$
	\begin{equation}
	\begin{split}
		9^{k+1} - 1 & = 9^k * 9 - 1 \\
		& = 9^k - 1 + 8*9^k
	\end{split}
	\end{equation}

	Da hipotese, temos que $9^k - 1$ é verdade, e $8*9^k$ é obviamente
	divisível por 8.
\end{proof}

\subsection*{(e)}
\newtheorem{teo5}{Teorema}
\begin{teo5}
	$n^2 - 1$ é divisível por 8 para qualquer $n \in \mathbb{N}$ ímpar
\end{teo5}

\begin{proof}
	Para $n = 1$, temos
	\begin{equation}
		\dfrac{1^2 - 1}{8} = 0
	\end{equation}

	Para $n = 3$, temos
	\begin{equation}
		\dfrac{3^2 - 1}{8} = \dfrac{8}{8} = 1
	\end{equation}

	Então vamos supor que para um $n = 2k + 1$ isso também dá certo.

	\begin{equation}
	\begin{split}
		(2k + 1)^2 - 1 & = 4k^2 + 4k + 1 - 1 \\
		& = 4k^2 + 4k \text{ é divisível por 8}
	\end{split}
	\end{equation}

	Daí vamos provar para $n = 2k + 3$
	\begin{equation}
	\begin{split}
		(2k + 3)^2 - 1 = 4k^2 + 12k + 9 - 1\\
		& = 4k^2 + 12k + 8\\
		& = 4k^2 + 4k + (8k - 8)\\
		& = 4k^2 + 4k + 8(k - 1)
	\end{split}
	\end{equation}

	Pela hipótese, $4k^2 + 4k$ é divisível por 8 e $8(k - 1)$ é obviamente
	divisível por 8.
\end{proof}

\subsection*{(f)}
\newtheorem{teo6}{Teorema}
\begin{teo6}
	$\sum\limits_{i = 1}^{n}i * (i+1) = \dfrac{n(n + 1)(n + 2)}{3} $
\end{teo6}

\begin{proof}
	Para $n = 1$, temos
	\begin{equation}
	\begin{split}
		1 * 2 & = \dfrac{1*(1 + 1)*(1 + 2)}{3} \\
		2 & = \dfrac{6}{3} = 2
	\end{split}
	\end{equation}

	Para $n = 2$ temos
	\begin{equation}
	\begin{split}
		(1 * 2)+(2 * 3) & = \dfrac{2*(2 + 1)(2 + 2)}{3}\\
		8 & = \dfrac{24}{3} = 8
	\end{split}
	\end{equation}

	Assumindo que seja verdade para $n = k$, 
	\begin{equation}
		\sum\limits_{i = 1}^{k}i * (i+1) = \dfrac{k(k + 1)(k + 2)}{3}
	\end{equation}


	vamos provar para $n = k +1$
	\begin{equation}
		\begin{split}
		\dfrac{(k + 1)(k + 2)(k + 3)}{3} & = \sum\limits_{i = 1}^{k+1}i * (i+1)\\
		&  = \sum\limits_{i = 1}^{k}i * (i+1) + (k + 1)(k + 2) \\
		&  = \dfrac{k(k + 1)(k + 2)}{3} + (k^2 + 3k + 2) \\
		&  = \dfrac{k(k + 1)(k + 2) + 3*(k^2 + 3k + 2) }{3}  \\
		&  = \dfrac{k^3 + 3k^2 + 2k + 3k^2 + 9k + 6 }{3}  \\
		&  = \dfrac{k^3 + 6k^2 + 11k + 6}{3}  \\
		&  = \dfrac{(k + 1)(k + 2)(k + 3)}{3} 
		\end{split}
	\end{equation}
\end{proof}

\subsection*{(g)}
\newtheorem{teo7}{Teorema}
\begin{teo7}
	$2^{n+2} + 3^{2n+1}$ é divisível por 7 para todo $n \in \mathbb{N^+}$
\end{teo7}
\begin{proof}
	Para $n = 0$
	\begin{equation}
		\dfrac{2^{0 + 2} + 3^{0 + 1}}{7} = \dfrac{4 + 3}{7} = 1
	\end{equation}

	Para $n = 1$ temos
	\begin{equation}
		\dfrac{2^{1+2}+3^{2+1}}{7} = \dfrac{8 + 27}{7} = 5 
	\end{equation}

	Vamos supor que para $n = k$ isso também é verdade
	\begin{equation}
		2^{k+2} + 3^{2k+1}\text{, é divisível por 7}
	\end{equation}

	Agora vamos provar que para $n = k + 1$ também é verdade
	\begin{equation}
	\begin{split}
		2^{k+3} + 3^{2(k+1)+1} & = 2^{k+2} + 2^{k+2} + 3^{2k + 3}\\
		& = 2^{k+2} + 2^{k+2} + 3^{2k+1} + 3^{2k+1} + 7*3^{2k + 1} \\
		& = (2^{k+2} + 3^{2k+1}) + (2^{k+2} + 3^{2k+1}) + 7*3^{2k + 1} 
	\end{split}
	\end{equation}

	Pela hipótese $2^{k+2} + 3^{2k+1}$ é divisível por 7, e obviamente 
	$7*3^{2k+1}$ também.
\end{proof}
\newpage

\section{Exercício 2}
\begin{equation}
	\sum\limits_{k=0}^{n}\left(\begin{array}{c} n \\ k \end{array} \right) = 2^n
\end{equation}


\begin{proof}
Para $n = 1$, temos
\begin{equation}
	\begin{split}
	\sum\limits_{k=0}^{1}\left(\begin{array}{c} 1 \\ k \end{array} \right) &= 2^1\\
	\left(\begin{array}{c}1 \\ 0 \end{array} \right)+\left(\begin{array}{c} 1 \\ 1 \end{array} \right) & = \\
	\left(\begin{array}{c} 0 \\ 1 \end{array} \right) + \left(\begin{array}{c} 0 \\ 0 \end{array} \right) & =\\
	1 + 1 &= 2
	\end{split}
\end{equation}

Para $n = 2$, temos
\begin{equation}
	\begin{split}
	\sum\limits_{k=0}^{2}\left(\begin{array}{c} 2 \\ k \end{array} \right) & = 2^2\\
	\left(\begin{array}{c} 2 \\ 0 \end{array} \right) + \left(\begin{array}{c} 2 \\ 1 \end{array} \right) + \left(\begin{array}{c} 2 \\ 2 \end{array} \right) & = \\
	\left(\begin{array}{c} 1 \\ 1 \end{array} \right) + \left(\begin{array}{c} 1 \\ 0 \end{array} \right) + \left(\begin{array}{c} 1 \\ 2 \end{array} \right) + \left(\begin{array}{c} 1 \\ 1 \end{array} \right) & =\\
	\left(\begin{array}{c} 0 \\ 1 \end{array} \right) + \left(\begin{array}{c} 0 \\ 0 \end{array} \right) + 0 + 0 + \left(\begin{array}{c} 0 \\ 1 \end{array} \right) + \left(\begin{array}{c} 0 \\ 0 \end{array} \right) &=\\
	1 + 1 + 1 + 1 & = \\
	4 & = 4
	\end{split}
\end{equation}

Vamos assumir que para $n = j$ isso é verdade
\begin{equation}
	\sum\limits_{k=0}^{j}\left(\begin{array}{c} j \\ k \end{array} \right) = 2^j
\end{equation}

Então, provaremos para $n = j + 1$
\begin{equation}
	\begin{split}
	\sum\limits_{k=0}^{j+1}\left(\begin{array}{c} j+1 \\ k \end{array} \right) & = 2^{j+1}\\
	\sum\limits_{k=0}^{j+1}\left(\begin{array}{c} j \\ k \end{array} \right) + \sum\limits_{k=0}^{j+1}\left(\begin{array}{c} j \\ k-1 \end{array} \right) & = \\
	\sum\limits_{k=0}^{j}\left(\begin{array}{c} j \\ k \end{array} \right) + \left(\begin{array}{c} j \\ j+1 \end{array} \right) + \left(\begin{array}{c} j \\ -1 \end{array} \right) + \left(\begin{array}{c} j \\ 0 \end{array} \right) + .. + \left(\begin{array}{c} j \\ j \end{array} \right)& = \\
	2^j + 0 + 0 + \left(\begin{array}{c} j \\ 0 \end{array} \right) + .. + \left(\begin{array}{c} j \\ j \end{array} \right) & =  \\
	2^j + \sum\limits_{k=0}^{j}\left(\begin{array}{c} j \\ k \end{array} \right) & = \\
	2^j + 2^j & = 2^{j+1}
	\end{split}
\end{equation}
\end{proof}
\end{document}
