\documentclass{article}

\renewcommand{\thesection}{}
\renewcommand{\thesubsection}{\arabic{section}.\arabic{subsection}}
\makeatletter
\def\@seccntformat#1{\csname #1ignore\expandafter\endcsname\csname the#1\endcsname\quad}
\let\sectionignore\@gobbletwo
\let\latex@numberline\numberline
\def\numberline#1{\if\relax#1\relax\else\latex@numberline{#1}\fi}
\makeatother



\usepackage[utf8]{inputenc}

\title{Lista de Exercícios 4}
\author{Gustavo Higuchi}
\date{\today}

\usepackage{natbib}
\usepackage{graphicx}
\usepackage{amssymb}
\usepackage{amsthm}
\usepackage{amsmath}
\usepackage{color}   %May be necessary if you want to color links
\usepackage[portuguese, ruled, linesnumbered]{algorithm2e}

\usepackage{mathtools}
\DeclarePairedDelimiter\ceil{\lceil}{\rceil}
\DeclarePairedDelimiter\floor{\lfloor}{\rfloor}

% usado para linkar cada section na tabela de conteúdo com a respectiva
% página no documento
\usepackage{hyperref}
\hypersetup{
    colorlinks,
    citecolor=black,
    filecolor=black,
    linkcolor=black,
    urlcolor=black,
    linktoc=all
}

%o começo do documento
\begin{document}

% compila o título
\maketitle

% compila a tabela de conteúdos
\tableofcontents
\newpage


\chapter{}
\textbf{assumindo que \'i até n\' seja equivalente à \'i <= n\' }
\section{Exercício 1}
\subsection*{(a)}
O algoritmo sempre retorna $n$, para qualquer $n>0$
\subsection*{(b)}
No pior caso e melhor caso, executa $O(n)$ operações

\section{Exercício 2}
\subsection*{(a)}
O algoritmo sempre retorna $n^2$, para qualquer $n>0$
\subsection*{(b)}
No pior caso e no melhor caso, executa $O(n^2)$ operações

\section{Exercício 3}
\subsection*{(a)}
O algoritmo retorna $\sum\limits_{i=0}^{n} i$
\subsection*{(b)}
O algoritmo roda em $O(n)$

\section{Exercício 4}
\subsection*{(a)}
O algoritmo retorna $2^n$
\subsection*{(b)}
O algoritmo roda em $O(n^3)$

\section{Exercício 5}
\subsection*{(a)}
O algoritmo retorna $n^3$
\subsection*{(b)}
O algoritmo roda em $O(n^3)$

\section{Exercício 6}
\subsection*{(a)}
O algoritmo retorna
\subsection*{(b)}
O algoritmo roda em $O(n^3)$

\section{Exercício 7}
\subsection*{(a)}
A operação fundamental do algoritmo é a linha 4.
\subsection*{(b)}
A linha 1 executa 1 vez\\
A linha 2 executa 1 vez\\
A linha 3 executa $n - 2$ vezes\\
A linha 4 executa $n - 2$ vezes\\
A linha 5 executa 1 vez\\
Assim, o algoritmo executa um total de $2n-1$ vezes
\subsection*{(c)}
O algoritmo executa em tempo $O(n)$

\section{Exercício 8}
\subsection*{(a)}
A operação fundamental é a linha 5
\subsection*{(b)}
A linha 1 executa 1 vez\\
A linha 2, 3, 4 e 5 executam $\log_c n$ vezes\\
Totalizando $4\log_c z + 1$ operações
\subsection*{(c)}
O algoritmo roda em $O(\log_c z)$

\section{Exercício 9}
\subsection*{(a)}
A operação fundamental são as linhas 3 e 4
\subsection*{(b)}
A linha 1 executa 1 vez\\
As linhas 2, 3 e 4 executam um total de $z$ vezes\\
Totalizando, fica $3z+1$ operações
\subsection*{(c)}
O algoritmo roda em $O(z)$

\section{Exercício 10}
\subsection*{(a)}
A operação fundamental é a linha 5
\subsection*{(b)}
A linha 1 executa $n-1$ vezes\\
A linha 2 executa $n-i$\\
A linha 3 executa $n-i$ vezes\\
Um total de $\dfrac{n^2-n}{2}$
\subsection*{(c)}
O algoritmo roda em $O(n^2)$

\end{document}
