\documentclass{article}

\renewcommand{\thesection}{}
\renewcommand{\thesubsection}{\arabic{section}.\arabic{subsection}}
\makeatletter
\def\@seccntformat#1{\csname #1ignore\expandafter\endcsname\csname the#1\endcsname\quad}
\let\sectionignore\@gobbletwo
\let\latex@numberline\numberline
\def\numberline#1{\if\relax#1\relax\else\latex@numberline{#1}\fi}
\makeatother



\usepackage[utf8]{inputenc}

\title{Lista de Exercícios 5}
\author{Gustavo Higuchi}
\date{\today}

\usepackage{natbib}
\usepackage{graphicx}
\usepackage{amssymb}
\usepackage{amsthm}
\usepackage{amsmath}
\usepackage{color}   %May be necessary if you want to color links

\usepackage{mathtools}
\DeclarePairedDelimiter\ceil{\lceil}{\rceil}
\DeclarePairedDelimiter\floor{\lfloor}{\rfloor}

% usado para linkar cada section na tabela de conteúdo com a respectiva
% página no documento
\usepackage{hyperref}
\hypersetup{
    colorlinks,
    citecolor=black,
    filecolor=black,
    linkcolor=black,
    urlcolor=black,
    linktoc=all  
}

%o começo do documento
\begin{document}

% compila o título
\maketitle

% compila a tabela de conteúdos
\tableofcontents
\newpage


\chapter{}

\section{Exercício 1}
    \textbf{Fatos dos algoritmo} \\
    \hspace*{30pt} $a_0 = 2$,\hspace{30pt} $m_0 = A[1]$ \\
    \hspace*{30pt} $a_{j+1} = a_j + 1$, $m_j+1 = max(m_{j}, A[j+1])$\\
    \textbf{Invariantes de laços} \\
    \\
    \textbf{Teorema.} Para todo natural $j \geq 0$, $i_j = j + 2$ e $m_j = max(m_{j-1}, A[j])$.
    
    \begin{proof}
        \hfill \break
        \textbf{Base:} \\
        \hspace*{30pt} Para $j = 0$ é trivial:\\
        \hspace*{30pt} $i_0 = 2$ e $m_0 = max(m_0, A[1]) = A[1]$\\
        \textbf{Hipotese:}\\
        \hspace*{30pt} Para $j \geq 0$, $i_j = j + 2$ e $m_j = max(m_{j-1}, A[j])$\\
        \textbf{Passo:}\\
        \hspace*{30pt} Queremos mostrar que $i_{j+1} = j + 3$ e $m_{j+1} = max(m_j, A[j+1])$\\
        \begin{equation}
            \begin{split}
                i_{j+1} & = i_j + 1\\
                \text{(hip.)} & = (j + 1) + 2\\
                & = j + 3\\
                \\
                m_{j+1} & = max(m_j, A[i])\\
            \end{split}
        \end{equation}
    \end{proof}
\end{document}
