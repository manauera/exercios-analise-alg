\documentclass{article}

\renewcommand{\thesection}{}
\renewcommand{\thesubsection}{\arabic{section}.\arabic{subsection}}
\makeatletter
\def\@seccntformat#1{\csname #1ignore\expandafter\endcsname\csname the#1\endcsname\quad}
\let\sectionignore\@gobbletwo
\let\latex@numberline\numberline
\def\numberline#1{\if\relax#1\relax\else\latex@numberline{#1}\fi}
\makeatother

\usepackage[utf8]{inputenc}

\title{Lista de Exercícios 2}
\author{Gustavo Higuchi}
\date{\today}

\usepackage{natbib}
\usepackage{graphicx}
\usepackage{amssymb}
\usepackage{amsthm}
\usepackage{amsmath}
\usepackage{color}   %May be necessary if you want to color links
\usepackage{al­go­rith­m2e} %preamble


% usado para linkar cada section na tabela de conteúdo com a respectiva
% página no documento
\usepackage{hyperref}
\hypersetup{
    colorlinks,
    citecolor=black,
    filecolor=black,
    linkcolor=black,
    urlcolor=black,
    linktoc=all  
}

%o começo do documento
\begin{document}

% compila o título
\maketitle

% compila a tabela de conteúdos
\tableofcontents
\newpage


\chapter{}

\section{Exercício 1}
\begin{proof}
	\hfill \break
	\textbf{Base} : para $n = 0$, g(n) devolve 0 e está correto,\newline
	\hspace*{30pt} para $n = 1$, g(n) devolve 1 e está correto também\newline
	\newline
	\textbf{Hipótese} : para $n \geq 2$ e todo $0\geq m < n$, g(m) devolve $3^m-2^m$\newline
	\textbf{Passo} : queremos provar que para g(n), o algoritmo devolve $3^n-2^n$.\newline
	\hspace*{30pt} assim temos \newline
	\begin{equation}
		\begin{split}
		& 5*g(n-1) - 6*g(n-2) \\
		&5*(3^{n-1} - 2^{n-1}) - 6(3^{n-2}-2^{n-2})\\
		&5*3^{n-1} - 5*2^{n-1} - 2*3*3^{n-2} + 3*2*2^{n-2}\\
		&5*3^{n-1} - 2*3^{n-1} - 5*2^{n-1} + 3*2^{n-1}\\
		&3*3^{n-1} - 2*2^{n-1} = 3^n - 2^n
		\end{split}	
	\end{equation}	

\end{proof}

\section{Exercício 2}
\begin{algorithm}
	\KwIn{entities; actors; player; customRules;}
	\KwOut{new world state}
	let \textit{entities} be all entities, excluding actors and the player\;
	let \textit{actors} be all actors, including the player\;
	\ForEach{actor in actors}{
		actor.rules.Invoke()\;
	}
	\tcc{It's possible that one of the rules gave the player a constraint, so we'll check}
	\If{player.constraint == null}{
		GenerateConstraint(player)\;
	}
	\ForEach{rule in customRules}{
		rule.Invoke()\;
	}
	\caption{Step planning}\label{alg:step_planning}
\end{algorithm}
\begin{proof}
	\hfill \break
	\textbf{Base} : para $n = 0$, g(n) devolve 0 e está correto,\newline
	\hspace*{30pt} para $n = 1$, g(n) devolve 1 e está correto também\newline
	\newline
	\textbf{Hipótese} : para $n \geq 2$ e todo $0\geq m < n$, g(m) devolve $3^m-2^m$\newline
	\textbf{Passo} : queremos provar que para g(n), o algoritmo devolve $3^n-2^n$.\newline
	\hspace*{30pt} assim temos \newline
	\begin{equation}
		\begin{split}
		& 5*g(n-1) - 6*g(n-2) \\
		&5*(3^{n-1} - 2^{n-1}) - 6(3^{n-2}-2^{n-2})\\
		&5*3^{n-1} - 5*2^{n-1} - 2*3*3^{n-2} + 3*2*2^{n-2}\\
		&5*3^{n-1} - 2*3^{n-1} - 5*2^{n-1} + 3*2^{n-1}\\
		&3*3^{n-1} - 2*2^{n-1} = 3^n - 2^n
		\end{split}	
	\end{equation}	
\end{proof}
\end{document}
