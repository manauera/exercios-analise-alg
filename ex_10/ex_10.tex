\documentclass{article}

\renewcommand{\thesection}{}
\renewcommand{\thesubsection}{\arabic{section}.\arabic{subsection}}
\makeatletter
\def\@seccntformat#1{\csname #1ignore\expandafter\endcsname\csname the#1\endcsname\quad}
\let\sectionignore\@gobbletwo
\let\latex@numberline\numberline
\def\numberline#1{\if\relax#1\relax\else\latex@numberline{#1}\fi}
\makeatother



\usepackage[utf8]{inputenc}

\title{Lista de Exercícios 4}
\author{Gustavo Higuchi}
\date{\today}

\usepackage{natbib}
\usepackage{graphicx}
\usepackage{amssymb}
\usepackage{amsthm}
\usepackage{amsmath}
\usepackage{color}   %May be necessary if you want to color links
\usepackage[portuguese, ruled, linesnumbered]{algorithm2e}

\usepackage{mathtools}
\DeclarePairedDelimiter\ceil{\lceil}{\rceil}
\DeclarePairedDelimiter\floor{\lfloor}{\rfloor}

% usado para linkar cada section na tabela de conteúdo com a respectiva
% página no documento
\usepackage{hyperref}
\hypersetup{
    colorlinks,
    citecolor=black,
    filecolor=black,
    linkcolor=black,
    urlcolor=black,
    linktoc=all
}

%o começo do documento
\begin{document}

% compila o título
\maketitle

% compila a tabela de conteúdos
\tableofcontents
\newpage


\chapter{}
\section{Exercício 1}
MergeSort([3,41,52,26,38,57,9,49])\\
chama recursivamente para MergeSort([3,41,52,26])\\
depois, MergeSort([3,41])\\
Então retorna para a execução anterior, e executa a outra parte:\\
MergeSort([52,26]), que retorna [26,52]\\
daí intercala e retorna [3,26,41,52]\\
\\
Voltando para a chamada inicial, chama recursivamente MergeSort([38,57,9,49])\\
Chama recursivamente MergeSort([38,57]) e retorna [38,57]\\
e executa a outra parte, MergeSort([9,49]), que retorna [9,49]\\
Retornando para a chamada inicial com [9,38,49,57], temos\\
Intercala([3,26,41,52], [9,38,49,57]), que retorna [3,9,26,38,41,49,52,57]
e termina a execução do algoritmo.

\section{Exercício 2}

\end{document}
